\documentclass{article}
\usepackage[utf8]{inputenc}

\title{Trabajo Practico 1}
\author{De Rosa - Schapira - Guerrero}
\date{Primer Cuatrimestre 2017}

\begin{document}
    \pagenumbering{gobble}
    \maketitle
    \newpage
    \pagenumbering{roman}
    \tableofcontents
    \newpage
    \pagenumbering{arabic}

    \section{Asignacion de residencias}
        \subsection{Objetivo}
            Solucionar el problema de la asignación de residencias utilizando
            el algoritmo de Gale-Shapley de Matching Estable.
        \subsection{Conclusiones}
            \subsubsection{Complejidad del algoritmo}
                Se prueba que el algoritmo tiene una complejidad $O(n(n-1) + 1)$ pero
                dado que es un análisis de peor caso, en la práctica se obtuvo un rendimiento
                mejor. \\
                Si se hacen pruebas con cantidades de estudiantes ($n$) y de hospitales
                ($m$) iguales se obtienen los siguientes resultados:
                %Tabla de resultados
                \begin{table}[h!]
                    \centering
                    \caption{Tiempo de resolución del problema}
                    \begin{tabular}{c|c|c}
                        n & m & t \\
                        \hline
                        10 & 10 & 0.6ms \\
                        \hline
                        100 & 100 & 30ms \\
                        \hline
                        500 & 500 & 2.5s \\
                        \hline
                        1000 & 1000 & 19.1s \\
                        \hline
                        3000 & 3000 & 8m 34s
                    \end{tabular}
                \end{table}

                Como se puede ver, estos tiempos representan valores mucho menores a
                los esperados por un algoritmo $O(n(n-1)+1)$.
            \subsubsection{Reduccion del problema de los casamientos}
                El problema de la asignación de residencias puede reducirse al problema
                de los casamientos en tiempo polinomial de la siguiente manera: \\
                Teniendo en cuenta que se dispone de una lista H de listas de preferencias
                donde en cada indice se encuentra la lista de preferencias del hospital
                representado por dicho indice, una lista E del mismo estilo pero para los
                estudiantes y una lista Q que en cada indice tiene un entero que representa
                las vacantes del hospital de dicho indice, podemos adaptar esta informacion
                a la necesaria para ser procesada por el algoritmo de formacion de parejas de
                Gale Shapley adaptando la lista H a una nueva H' con la información de Q. \\
                La manera de hacer esto sería: \\
                H' contendrá las listas de preferencias de H pero repetidas Q[i] veces. Es decir,
                si H[0] tenia 3 vacantes, entonces en H', H[0], H[1] y H[2] tendrán las mismas preferencias
                y el hospital 0 pasará a ser representado por el conjunto {0, 1, 2}. \\
                Dicho esto, ahora habría que crear una lista E' donde, para cada lista de
                preferencias E[i], existirá una lista E'[i] donde para cada hospital k, se
                pondrá el conjunto k = {p, q, r}. En el ejemplo anterior, si E[i] = [0], entonces
                E'[i] = [0, 1, 2]. \\
                Con estas listas inicializadas, se corre Gale Shapley que devolverá parejas para
                cada hospital de H'. Es decir, P = [(H'i, E'k), ..., (H'n, E'j)]. Donde solo len(E) estudiantes
                serán no NULL. Ahora, como ya conocemos en que H'k se convirtió cada Hi, podemos agrupar
                los E' de las parejas devueltas por Gale Shapley para finalizar el algoritmo y resolver el
                problema de asignación de residencias. Es decir, con el ejemplo anterior, pondremos para el
                hospital 0, los estudiantes asignados por Gale Shapley a los hospitales H'[0], H'[1] y H'[2]. \\
                Si analizamos la complejidad, podemos ver lo siguiente: \\
                Crear la lista H' es $O(m*q)$, siendo m la cantidad de estudiantes y q la sumatoria de todas las vacantes,
                pues hay q listas de m elementos (antes eran n pero ahora se repiten q veces). \\
                Crear la lista E' es $O(m*q)$, siendo m y q los mismos de antes, pues hay m listas de q elementos. \\
                Gale Shapley es polinomial. \\
                Adaptar el resultado es $O(q)$, pues se debe recorrer una lista de q parejas, añadiendo el estudiante
                de cada pareja al hospital que corresponda según los cambios hechos al principio.

    \section{Puntos de Falla}
        \subsection{Objetivo}
            Encontrar los puntos de fallas (puntos de articulación) de una red eléctrica
            mediante el algoritmo de Hopcroft y Tarjan.
        \subsection{Conclusiones}
            \subsubsection{Funcionamiento del algoritmo}
                El algoritmo comienza tomando un vértice cualquiera del grafo, a partir
                del cual se realiza un recorrido dfs. Este genera un grafo del recorrido
                iniciado en el vértice elegido. En este grafo, será punto de articulación
                cada vértice que tenga dos o mas conocidos, es decir que tenga dos o mas
                hijos. Luego para cada vértice en el grafo de recorrido dfs, chequeamos
                la cantidad de vecinos que tiene. Si se cumple la condición, entonces es un punto
                de articulación. Por lo tanto el orden del algoritmo será $O(v+e)$.

            \subsubsection{Tiempo de ejecucion}
                Haciendo la prueba con una cantidad $(v)$ de vértices obtenemos lo siguiente:
                %Tabla de resultados
                \begin{table}[h]
                    \centering
                    \caption{Tiempo de resolución del problema}
                    \begin{tabular}{c|c}
                        v & t \\
                        \hline
                        10 & 0.25 ms \\
                        \hline
                        100 & 1.8 ms \\
                        \hline
                        1000 & 9.5 ms \\
                        \hline
                        10000 & 0.093 s \\
                        \hline
                        100000 & 0.99 s \\
                        \hline
                        1000000 & 10.34 s
                    \end{tabular}
                \end{table}

                Como se puede observar, los tiempos aumentan linealmente a medida
                que aumenta la cantidad de vértices. Esto condice con los tiempos
                esperados del algoritmo.


    \section{Comunidades en Redes}
        \subsection{Objetivo}
          Encontrar las componentes fuertemente conexas de las comunidades en las
          redes implementando el algoritmo de kosaraju.
        \subsection{Conclusiones}
            \subsubsection{Funcionamiento del algoritmo}
                El algoritmo comienza realizando un recorrido dfs de cada uno de los
                vértices. Esto se realiza para ordenar los vértices en sentido
                decreciente de acuerdo a su tiempo de finalización entre todos los
                recorridos dfs. Una vez que se obtiene el orden de los vértices,
                se invierte el grafo, es decir que se invierten todas sus aristas.
                El tiempo que toma invertir el grafo es $O(v+e)$. Luego se vuelve a
                realizar un recorrido dfs para cada vértice iniciando por el vértice
                que tiene mayor tiempo de finalización. Cada grafo que devuelto por cada
                recorrido dfs es una componente fuertemente conexa.
                El orden del algoritmo será $O(v+e)$

            \subsubsection{Tiempo de ejecucion}
              Haciendo la prueba con una cantidad v de vértices y a de arístas obtenemos:
              %Tabla de resultados
                \begin{table}[h!]
                    \centering
                    \caption{Tiempo de resolución del problema}
                    \begin{tabular}{c|c|c}
                        v & a & t \\
                        \hline
                        10 & 20 & 0.00084 s \\
                        \hline
                        100 & 250 & 0.048 s \\
                        \hline
                        1000 & 2500 & 3.008 s \\
                        \hline
                        10000 & 25000 & 322.01 s
                    \end{tabular}
                \end{table}

                Como podemos observar, los tiempos de resolución aumentan en medida
                cuadrática. Esto no concuerda con la predicción esperada, la cuál es
                lineal. Atribuimos esto a la eficiencia de la computadora en la que
                se hicieron las pruebas.

    \section{Comandos}
        Para correr los archivos, entrar en la carpeta correspondiente al algoritmo y ejecutar: \\
        python matchingEstable.py \\
        python puntosDeArticulacion.py \\
        python componentesConexas.py \\ \\
        ATENCIÓN: Se deben agregar los archivos de prueba para puntos de articulacion y componentes
        conexas. No se enviaron debido al peso que soporta gmail.

\end{document}
