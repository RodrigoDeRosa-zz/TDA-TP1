\documentclass{article}

\title{Trabajo Practico 1}
\author{De Rosa - Schapira - Guerrero}
\date{Primer Cuatrimestre 2017}

\begin{document}
    \pagenumbering{gobble}
    \maketitle
    \newpage
    \pagenumbering{roman}
    \tableofcontents
    \newpage
    \pagenumbering{arabic}

    \section{Asignacion de residencias}
        \subsection{Objetivo}
            Solucionar el problema de la asignacion de residencias utilizando
            el algoritmo de Gale-Shapley de Matching Estable.
        \subsection{Conclusiones}
            \subsubsection{Complejidad del algoritmo}
                Si bien el ciclo principal del algoritmo (en el que se resuelve el
                problema como si) tiene un orden de complejidad $O(nk)$, donde k
                es la cantidad maxima de vacantes de cada hospital, el algoritmo
                tiene un orden de complejidad $O(n^2)$ dado por la creacion de la
                matriz de preferencias, pues es necesario analizar la posicion de
                cada estudiante en la lista de cada uno de los hospitales.
            \subsubsection{Tiempo de ejecucion}
                Si se hacen pruebas con cantidades de estudiantes ($n$) y de hospitales
                ($m$) iguales se obtienen los siguientes resultados:
                %Tabla de resultados
                \begin{table}[h!]
                    \centering
                    \caption{Tiempo de resolucion del problema}
                    \begin{tabular}{c|c|c}
                        n & m & t \\
                        \hline
                        10 & 10 & 0.6ms \\
                        \hline
                        100 & 100 & 30ms \\
                        \hline
                        500 & 500 & 2.5s \\
                        \hline
                        1000 & 1000 & 19.1s \\
                        \hline
                        3000 & 3000 & 8m 34s
                    \end{tabular}
                \end{table}

                Como se puede ver, estos tiempos representan valores mucho menores a
                los esperados por un algoritmo $O(n^2)$. Consideramos que esto es
                resultado de la forma en que se crea la anteriormente mencionada
                matriz de preferencias. Suponemos que la implementacion de los
                diccionarios por comprension de Python permiten un rendimiento mucho
                mejor que $O(n^2)$.
            \subsubsection{Reduccion del problema de los casamientos}
                El algoritmo implementado permite resolver el problema de matching
                estable cuando el grupo de \emph{reviewers} puede aceptar a mas de
                un \emph{aplicante}. Por lo tanto, si consideraramos que cada reviewer
                puede aceptar solo a \underline{un} aplicante, tenemos el problema
                de la formacion de parejas. Por lo tanto, vemos que ese problema
                puede ser reducido al ya resuelto si la lista Q de vacantes es tal
                que $Q = [1]*n$.

\end{document}
